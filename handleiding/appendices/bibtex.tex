\startcomponent _symbolen
\product        handleiding
\environment    handleidinglayout
% De tabel in dit bestand is overgenomen van http://nl.wikipedia.org/wiki/BibTeX en http://en.wikipedia.org/wiki/BibTeX.

\chapter[app:bibtex]{Bib\TeX\ documenttypes}

\startemphasize
Hieronder vind je een overzicht van alle \emph{\en entry types} van \BIBTEX\ met een omschrijving van het type. Ook zijn alle verplichte en optionele velden aangegeven.
\stopemphasize

\startitemize[unpacked]
\head @article@

Artikel in een wetenschappelijk tijdschrift.

Verplicht: @author@, @title@, @journal@, @year@

Optioneel: @volume@, @number@, @pages@, @month@, @note@, @key@

\head @book@

Een boek met een uitgever.

Verplicht: @author/editor@, @title@, @publisher@, @year@

Optioneel: @volume@, @series@, @address@, @edition@, @month@, @note@, @key@

\head @booklet@

Een gebundeld werk zonder uitgever.

Verplicht: @title@

Optioneel: @author@, @howpublished@, @address@, @month@, @year@, @note@, @key@

\head @electronic@

Webpagina.

Verplicht: @url@

Optioneel: @author@, @title@

\head @inbook@

Deel van een boek, bijvoorbeeld een hoofdstuk.

Verplicht: @author/editor@, @title@, @chapter/pages@, @publisher@, @year@

Optioneel: @volume@, @series@, @address@, @edition@, @month@, @note@, @key@

\head @incollection@

Deel van een boek met eigen titel.

Verplicht: @author@, @title@, @booktitle@, @publisher@, @year@

Optioneel: @editor@, @pages@, @organization@, @publisher@, @address@, @month@, @note@, @key@

\head @inproceedings@

Artikel in een conferentiebundel.

Verplicht: @author@, @title@, @booktitle@, @year@

Optioneel: @editor@, @series@, @pages@, @organization@, @publisher@, @address@, @month@, @note@, @key@

\head @manual@

Technische handleiding.

Verplicht: @title@

Optioneel: @author@, @organization@, @address@, @edition@, @month@, @year@, @note@, @key@

\head @mastersthesis@

Afstudeerverslag.

Verplicht: @author@, @title@, @school@, @year@

Optioneel: @address@, @month@, @note@, @key@

\head @misc@

Alles wat niet in een andere type past.

Verplicht: \emph{geen}

Optioneel: @author@, @title@, @howpublished@, @month@, @year@, @note@, @key@

\head @phdthesis@

Dissertatie.

Verplicht: @author@, @title@, @school@, @year@

Optioneel: @address@, @month@, @note@, @key@

\head @proceedings@

Conferentiebundel.

Verplicht: @title@, @year@

Optioneel: @editor@, @publisher@, @organization@, @address@, @month@, @note@, @key@

\head @techreport@

Onderzoeksrapport door een instituut.

Verplicht: @author@, @title@, @institution@, @year@

Optioneel: @type@, @number@, @address@, @month@, @note@, @key@

\head @unpublished@

Niet formeel gepubliceerd.

Verplicht: @author@, @title@, @note@

Optioneel: @month@, @year@, @key@
\stopitemize

\stopcomponent
