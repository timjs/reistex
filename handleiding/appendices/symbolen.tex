\startcomponent symbolen
\product        handleiding
\environment    handleidinglayout
% De tabellen in dit bestand zijn overgenomen uit ma-cb-en-math.tex en ma-cb-en-mathcharacters.tex van http://context.aanhet.net/svn/contextman/context-beginners/, waar aangegeven met aanpassingen.

% Super handige definitie om onderstaande tabellen een stuk overzichtelijker te houden:
\define[1]\SC
  {\NC \math{#1} \NC @#1@ }

\chapter[app:symbolen]{Wiskundige symbolen}

\startemphasize
Hier vind je een overzicht van veelgebruikte wiskundige symbolen en hun commando's in \TEX. Het zijn er aardig wat, maar lang niet allemaal! Ben je op zoek naar een specifiek symbool? Kijk dan eens op \from[detexify]. Daar kun je een symbool tekenen, waarna de site het juiste \TEX-commando voor je op zoekt.\footnote{Het kan zijn dat een symbool dat je op \from[detexify] vindt niet in \CONTEXT\ beschikbaar is. Mail de cie dan even.}
\stopemphasize

\section{Griekse letters}

\starttabulate[*3{|lm|l|p}|]
\NC        \SC {\alpha}      \NC         \SC {\iota}    \NC          \SC {\varrho}   \NC\NR
\NC        \SC {\beta}       \NC         \SC {\kappa}   \NC \Sigma   \SC {\sigma}    \NC\NR
\NC \Gamma \SC {\gamma}      \NC \Lambda \SC {\lambda}  \NC          \SC {\varsigma} \NC\NR
\NC \Delta \SC {\delta}      \NC         \SC {\mu}      \NC          \SC {\tau}      \NC\NR
\NC        \SC {\epsilon}    \NC         \SC {\nu}      \NC \Upsilon \SC {\upsilon}  \NC\NR
\NC        \SC {\varepsilon} \NC \Xi     \SC {\xi}      \NC \Phi     \SC {\phi}      \NC\NR
\NC        \SC {\zeta}       \NC         \SC {\omicron} \NC          \SC {\varphi}   \NC\NR
\NC        \SC {\eta}        \NC \Pi     \SC {\pi}      \NC          \SC {\chi}      \NC\NR
\NC \Theta \SC {\theta}      \NC         \SC {\varpi}   \NC \Psi     \SC {\psi}      \NC\NR
\NC        \SC {\vartheta}   \NC         \SC {\rho}     \NC \Omega   \SC {\omega}    \NC\NR
\stoptabulate

\section{Verzamelingen}

\starttabulate[*3{|l|p}|]% Toegevoegd
\SC {\naturalnumbers} \SC {\integers}  \SC {\rationals} \NC\NR
\SC {\reals}          \SC {\complexes} \SC {\primes}    \NC\NR
\stoptabulate

\section{Symbolen van variable grootte}

\starttabulate[*3{|l|p}|]% Herordend
\SC {\sum}     \SC {\prod}      \SC {\coprod}   \NC\NR
\SC {\int}     \SC {\oint}      \SC {\iint}     \NC\NR
\SC {\bigcup}  \SC {\bigcap}    \SC {\biguplus} \NC\NR
\SC {\bigodot} \SC {\bigotimes} \SC {\bigoplus} \NC\NR
\SC {\bigvee}  \SC {\bigwedge}  \SC {\bigsqcup} \NC\NR
\stoptabulate

\section{Standaardfuncties}

\starttabulate[*8{|pTb{\letterbackslash}}|]
\NC arccos \NC cos  \NC csc \NC exp \NC ker    \NC limsup \NC min \NC sinh \NC \NR
\NC arcsin \NC cosh \NC deg \NC gcd \NC lg     \NC ln     \NC Pr  \NC sup  \NC \NR
\NC arctan \NC cot  \NC det \NC hom \NC lim    \NC log    \NC sec \NC tan  \NC \NR
\NC arg    \NC coth \NC dim \NC inf \NC liminf \NC max    \NC sin \NC tanh \NC \NR
\stoptabulate

\section{Haakjes aanvullend op $()$, ${}$ en $[]$}

%FIXME: Hier werkt ons trucje helaas niet, er staat een gewone / tussen!
\starttabulate[*4{|l|p}|]% Herordend
\SC {\langle}    \SC {\vert}    \SC {\downarrow}   \SC {\lfloor} \NC\NR
\SC {\rangle}    \SC {\Vert}    \SC {\Downarrow}   \SC {\rfloor} \NC\NR
\SC {/}          \SC {\uparrow} \SC {\updownarrow} \SC {\lceil}  \NC\NR
\SC {\backslash} \SC {\Uparrow} \SC {\Updownarrow} \SC {\rceil}  \NC\NR
\stoptabulate

\section{Speciale tekens}

\starttabulate[*3{|l|p}|]
\SC {\aleph}   \SC {\prime}     \SC {\forall}      \NC\NR
\SC {\hbar}    \SC {\emptyset}  \SC {\exists}      \NC\NR
\SC {\imath}   \SC {\nabla}     \SC {\neg}         \NC\NR
\SC {\jmath}   \SC {\surd}      \SC {\flat}        \NC\NR
\SC {\ell}     \SC {\top}       \SC {\natural}     \NC\NR
\SC {\wp}      \SC {\bot}       \SC {\sharp}       \NC\NR
\SC {\Re}      \SC {\Vert}      \SC {\clubsuit}    \NC\NR
\SC {\Im}      \SC {\angle}     \SC {\diamondsuit} \NC\NR
\SC {\partial} \SC {\triangle}  \SC {\heartsuit}   \NC\NR
\SC {\infty}   \SC {\backslash} \SC {\spadesuit}   \NC\NR
\stoptabulate

\section{Operatoren aanvullend op $+$, $-$ en $*$}

\starttabulate[*3{|l|p}|]
\SC {\pm}       \SC {\cap}             \SC {\vee}      \NC\NR
\SC {\mp}       \SC {\cup}             \SC {\wedge}    \NC\NR
\SC {\setminus} \SC {\uplus}           \SC {\oplus}    \NC\NR
\SC {\cdot}     \SC {\sqcap}           \SC {\ominus}   \NC\NR
\SC {\times}    \SC {\sqcup}           \SC {\otimes}   \NC\NR
\SC {\ast}      \SC {\triangleleft}    \SC {\oslash}   \NC\NR
\SC {\star}     \SC {\triangleright}   \SC {\odot}     \NC\NR
\SC {\diamond}  \SC {\wr}              \SC {\dagger}   \NC\NR
\SC {\circ}     \SC {\bigcirc}         \SC {\ddagger}  \NC\NR
\SC {\bullet}   \SC {\bigtriangleup}   \SC {\amalg}    \NC\NR
\SC {\div}      \SC {\bigtriangledown} \SC {\smallint} \NC\NR
\stoptabulate

\section{Relaties aanvullend op $>$, $<$ en $=$}

\starttabulate[*3{|l|p}|]
\SC {\leq}        \SC {\geq}        \SC {\equiv}  \NC\NR
\SC {\prec}       \SC {\succ}       \SC {\sim}    \NC\NR
\SC {\preceq}     \SC {\succeq}     \SC {\simeq}  \NC\NR
\SC {\ll}         \SC {\gg}         \SC {\asymp}  \NC\NR
\SC {\subset}     \SC {\supset}     \SC {\approx} \NC\NR
\SC {\subseteq}   \SC {\supseteq}   \SC {\cong}   \NC\NR
\SC {\sqsubseteq} \SC {\sqsupseteq} \SC {\bowtie} \NC\NR
\SC {\in}         \SC {\ni}         \SC {\propto} \NC\NR
\SC {\vdash}      \SC {\dashv}      \SC {\models} \NC\NR
\SC {\smile}      \SC {\mid}        \SC {\doteq}  \NC\NR
\SC {\frown}      \SC {\parallel}   \SC {\perp}   \NC\NR
\stoptabulate

\section{Ontkennende relaties}

\starttabulate[*3{|l|p}|]
\SC {\not<}           \SC {\not>}           \SC {\not=}       \NC\NR
\SC {\not\leq}        \SC {\not\geq}        \SC {\not\equiv}  \NC\NR
\SC {\not\prec}       \SC {\not\succ}       \SC {\not\sim}    \NC\NR
\SC {\not\preceq}     \SC {\not\succeq}     \SC {\not\simeq}  \NC\NR
\SC {\not\subset}     \SC {\not\supset}     \SC {\not\approx} \NC\NR
\SC {\not\subseteq}   \SC {\not\supseteq}   \SC {\not\cong}   \NC\NR
\SC {\not\sqsubseteq} \SC {\not\sqsupseteq} \SC {\not\asymp}  \NC\NR
\stoptabulate

\section{Pijlen}

\starttabulate[*3{|l|l}|]%FIXME: tweede kolom net iets te lang voor |p|
\SC {\leftarrow}        \SC {\longleftarrow}      \SC {\uparrow}        \NC\NR
\SC {\Leftarrow}        \SC {\Longleftarrow}      \SC {\Uparrow}        \NC\NR
\SC {\rightarrow}       \SC {\longrightarrow}     \SC {\downarrow}      \NC\NR
\SC {\Rightarrow}       \SC {\Longrightarrow}     \SC {\Downarrow}      \NC\NR
\SC {\leftrightarrow}   \SC {\longleftrightarrow} \SC {\updownarrow}    \NC\NR
\SC {\Leftrightarrow}   \SC {\Longleftrightarrow} \SC {\Updownarrow}    \NC\NR
%\SC {\leftleftarrows}   \SC {\leftrightarrows}    \SC {\upuparrows}     \NC\NR
%\SC {\rightrightarrows} \SC {\rightleftarrows}    \SC {\downdownarrows} \NC\NR
\SC {\mapsto}           \SC {\longmapsto}         \SC {\nearrow}        \NC\NR
\SC {\hookleftarrow}    \SC {\hookrightarrow}     \SC {\searrow}        \NC\NR
\SC {\leftharpoonup}    \SC {\rightharpoonup}     \SC {\swarrow}        \NC\NR
\SC {\leftharpoondown}  \SC {\rightharpoondown}   \SC {\nwarrow}        \NC\NR
\stoptabulate

%\starttabulate[*3{|l|p}|]% Toegevoegd
%\SC {\twoheadleftarrow}  \SC {\leftleftarrows}   \SC {\upuparrows}     \NC\NR
%\SC {\leftarrowtail}     \SC {\leftrightarrows}  \SC {\Lsh}            \NC\NR
%\SC {\twoheadrightarrow} \SC {\rightrightarrows} \SC {\downdownarrows} \NC\NR
%\SC {\rightarrowtail}    \SC {\rightleftarrows}  \SC {\Rsh}            \NC\NR
%\stoptabulate
%\SC {\leadsto}

\section{Puntjes}

\starttabulate[*4{|l|p}|]% Toegevoegd
\SC {\cdots} \SC {\ldots} \SC {\vdots} \SC {\ddots} \NC\NR
\stoptabulate

\section{Accenten}

\starttabulate[*4{|l|p}|]% Toegevoegd
\SC {\acute{a}} \SC {\dot{a}}      \SC {\check{a}}   \SC {\bar{a}}       \NC\NR
\SC {\grave{a}} \SC {\ddot{a}}     \SC {\hat{a}}     \SC {\tilde{a}}     \NC\NR
\SC {\breve{a}} \SC {\mathring{a}} \SC {\widehat{a}} \SC {\widetilde{a}} \NC\NR
\SC {\vec{a}}   \SC {}             \SC {}            \SC {}              \NC\NR
\stoptabulate

\section{Alternatieve commando's}

\starttabulate[*5{|l|p}|]% Aangepast
\SC {\eq} \SC {\ne} \SC {\land} \SC {\to}   \SC {\implies}     \NC\NR
\SC {\lt} \SC {\le} \SC {\lor}  \SC {\gets} \SC {\iff}         \NC\NR
\SC {\gt} \SC {\ge} \SC {\lnot} \SC {\owns} \SC {\colonequals} \NC\NR
\stoptabulate

\stopcomponent

% vim: ft=context spell spl=nl cole=1
