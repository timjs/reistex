\startenvironment handleidinglayout
\enablemode[electronic]
\environment travellayout

\mainlanguage[nl]

% ------------------------------------------------------------------------------
% Definities
% ------------------------------------------------------------------------------

\starttexdefinition seedocumentation #1#2
  \inmargin{\from[excursion] §#1\crlf
            \from[manual] §#2}
\stoptexdefinition

\definetabulate[argumenttable]
  [|lTb{.}a{.}|lT|l|]
\definedescription[errormessage]
  [location=serried,
   width=fit,
   headstyle=\tt]

% Niet \makecharacteractive gebruiken, dan vergeet ie 'm weer,
% nu passen we \ctxcatcodes impliciet aan:
%\installactivecharacter @ %@
% Slim, maar werkt alleen in klein voorbeeld:
%\defineactivecharacter @
%  {\type@}
% Pakt 'm tot de laatste @:
%\def@#1@
%  {\rescan{\type{#1}}}
% Werkt niet met \type:
%\defineactivecharacter @ %
%  {\bgroup\tt
%   \defineactivecharacter @ {\egroup}}

% ------------------------------------------------------------------------------
% Instellingen
% ------------------------------------------------------------------------------

\setupheads
  [sectionnumber=yes]

\setupcombinedlist[content]
  [level=subsection,
   headnumber=yes]
\setuplist[part]
  [headnumber=no,
   after={\blank[big]}]
\setuplist[chapter]
  [before=\blank,
   style=\ss\bf,
   pagenumber=no,
   margin=0em,
   width=2em]
\setuplist[section]
  [margin=2em,
   width=3em]
\setuplist[subsection]
  [margin=5em,
   width=3em]

%\setupfootertexts
  %[][{\getmarking[chapter] \quad \pagenumber}]
  %[{\pagenumber \quad Reis\TeX en voor iedereen}][]

\setuptyping
  [palet=graypretty,
   option=TEX]
%\setuptype %FIXME: error bij \chemical{}
  %[palet=graypretty,
   %option=TEX]

\setupinteraction
  [title={Reis\TeX en voor iedereen},
   author={Tim Steenvoorden},
   subtitle={Handleiding voor het gebruik van ConTeXt bij verslagen van de SMCR},
   keyword={ConTeXt, TeX, SMCR, Marie Curie, reizen, verslag, reisverslag, reader, draaiboek, commando, commando's}]

\setupsectionblock[backpart]
  [page=no]% Zodat we geen extra pagina aan het einde krijgen.

% ------------------------------------------------------------------------------
% Url's TODO: BibTeX gebruiken
% ------------------------------------------------------------------------------

\useurl[excursion]
  [http://www.pragma-ade.com/general/manuals/ms-cb-en.pdf][]
  [Excursion]
\useurl[manual]
  [http://www.pragma-ade.com/general/manuals/cont-eni.pdf][]
  [Manual]
\useurl[wiki]
  [http://wiki.contextgarden.net][]
  [Wiki]
\useurl[commands]
  [http://wiki.contextgarden.net/Category:Reference/en][]
  [Command Reference]
\useurl[mathalign]
  [http://dl.contextgarden.net/myway/mathalign.pdf][]
  [Using \tex{startalign} and friends]
\useurl[tabulating]
  [http://www.ntg.nl/maps/22/28.pdf][]
  [Tabulating in \CONTEXT]
\useurl[chemical]
  [http://www.pragma-ade.com/general/manuals/mp-ch-en.pdf][]
  [\PPCHTEX]
\useurl[eenheden]
  [http://www.ntg.nl/maps/21/16.pdf][]
  [Eenheid in Eenheden]
\useurl[bib]
  [http://modules.contextgarden.net/bibman][]
  [Bibliography documentation]
\useurl[detexify]
  [http://detexify.kirelabs.org/classify.html][]
  [De\TEX ify]

% ------------------------------------------------------------------------------
% Logo's en afkortingen
% ------------------------------------------------------------------------------

\startbuffer[example:abbreviations]
\abbreviation{MC}   {Studievereniging Marie Curie}
\abbreviation{FLARE}{Free-electron Laser for Advanced spectroscopy
                     and high-Resolution Experiments}
\stopbuffer
\getbuffer[example:abbreviations]

\stopenvironment

% vim: ft=context spell spl=nl cole=1
