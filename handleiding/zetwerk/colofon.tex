\startcomponent _colofon
\product handleiding
\environment handleidinglayout

\startstandardmakeup

\vfill

\subject{Dankwoord}

Mijn dank gaat uit naar Maaike Zwart en Karel Kok, die de voorloper van \in{hoofdstukken}[chp:schrijven] \in{en}[chp:compileren] hebben geschreven. Dankzij hun wist ik wat de meest gebruikte functies van \TEX\ zijn. Maartje A.B. wil ik bedanken voor haar opbouwende kritiek, onuitputtelijke enthousiasme en haar scherpe oog voor fouten en inconsistenties. Dankzij Eline de Jong zijn er aanvullende commando's geboren, die onmisbaar zijn bij het zetten van een reisverslag.

\blank

Tot slot wil ik de deelnemers van de Denemarkenreis~2011 en de Chinareis~2011 bedanken voor hun geduld en medewerking bij het schrijven van hun stukjes. Zij waren de proefkonijnen en het botsingsexperiment die een nieuwe werkwijze en deze handleiding moesten doorstaan.

\subject{Opmaak}

Deze handleiding is geproduceerd met \CONTEXT\ versie \contextversion, gebruikmakend van \texenginename\ en \METAPOST.

\blank

De broodtekst is gezet in Utopia 11/13 $\times$ 34, ontworpen door Robert Slimbach in 1989.  
Voor de bijschriften en de koppen is gebruik gemaakt van Helvetica, in 1957 ontworpen door Max Miedinger en Eduard Hoffmann.

\blank

Het tekstblok voldoet aan de Gulden Ratio, evenals alle marges. Hierbij zijn zowel rug- en kopwit als snij- en voetwit gelijk genomen. Vrij naar een voorstel van Robert Bringhurst (\emph{\en The Elements of Typographic Style}, p.~175).

\stopstandardmakeup

\stopcomponent

% vim: spell spelllang=nl
