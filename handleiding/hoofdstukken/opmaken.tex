\startcomponent opmaken
\product        handleiding
\environment    handleidinglayout

\chapter[chp:opmaak,chp:opmaken]{Tips voor de opmaak}

\startemphasize
In dit hoofdstuk geven we enkele tips voor het opmaken van een verslag, reader of ander document. We behandelen uitgebreid het plaatsen van figuren. Iets waar \LATEX\ nogal eens moeite mee heeft, maar de makers van \CONTEXT\ extra aandacht aan besteed hebben. Daarnaast behandelen we nog enkele commando's die speciaal voor het zetten van verslagen en readers zijn toegevoegd aan \type{travellayout.tex}.
\stopemphasize

\placecontent

\section{De omgeving}

De stijl van alle stukjes, het verslag, de reader en ook dit document staat gedefinieerd in \type{travellayout.tex}. Vandaar dat we deze telkens boven aan het bronbestand aanroepen met
\starttyping
\environment travellayout.tex
\stoptyping
In \CONTEXT-jargon noemen we dit de \emph{omgeving} waarin we het document zetten. Dit zorgt ervoor dat we structuur en opmaak gescheiden kunnen houden.

Met stijl bedoelen we de paginamarges, opmaak van de kopjes en inhoudsopgave, maar ook de voetnoten, bijschriften van figuren en tabellen, de paginanummering enzovoorts. In \type{travellayout.tex} is voldoende uitleg te vinden over de gebruikte commando's. Dit commentaar kun je natuurlijk in het bestand zelf lezen. Maar \CONTEXT\ kan het geheel ook in \PDF-vorm voor je uitdraaien. Dit doe je door de omgeving te compileren met de optie \type{--module}.
\starttyping
texexec --module travellayout
\stoptyping

\section{Toegevoegde commando's}

In de componenten en producten gebruiken we eigenlijk alleen maar standaard \CONTEXT\ commando's. Dat betekend dat we de stukjes ook gewoon kunnen compileren zonder de omgeving in elk bronbestand aan te roepen of door domweg het bestand \type{travellayout.tex} weg te laten uit de mapstructuur.

Hier zijn twee belangrijke uitzonderingen op. Dit zijn de commando's \tex{emph} en \tex{report}. De eerste is gedefinieerd voor het gemak. Het is een standaard commando in \LATEX\ en leest veel fijner dan de \CONTEXT-declaratie \type{{\em }}. De tweede is een wrapper om \tex{chapter}. Het zorgt ervoor dat we ook een auteur en een datum kunnen toekennen aan ieder hoofdstuk.

\subsection{\tex{emph} en \tex{start/stopemphasize}}

\startframedtext
\startsidebysidetext
Voor woorden:
\starttyping
\emph {.1.}
\stoptyping
\nextsidebysidetext
Voor alinea's:
\starttyping
\startemphasize
.1.
\stopemphasize
\stoptyping
\stopsidebysidetext
\startargumenttable
\NC  1  \NC  tekst     \NC  tekst om te benadrukken  \NR
\stopargumenttable
\stopframedtext

\tex{emph} zorgt er voor dat we tekst op een gestructureerde manier kunnen benadrukken, zonder direct een \emph{stijl} toe te kennen, zoals vet, klein kapitaal of cursief\seedocumentation{31.7}{5.6}. Daarbij is het commando slim genoeg om bijvoorbeeld benadrukte tekst binnen verschillende stijlen goed weer te geven zoals te zien in het volgende voorbeeld.

\startbuffer
De afkorting \emph{emph} staat voor \emph{emphasis}.
\emph{De afkorting \emph{emph} staat voor \emph{emphasis}.}
{\bf De afkorting \emph{emph} staat voor \emph{emphasis}.}
\emph{\bf De afkorting \emph{emph} staat voor \emph{emphasis}.}
{\it De afkorting \emph{emph} staat voor \emph{emphasis}.}
\stopbuffer
\typebuffer

\startlines
\getbuffer
\stoplines

Om meerdere alinea's te benadrukken kun je ze tussen \tex{startemphasize} en \tex{stopemphasize} zetten.

\subsection[sec:report]{\tex{report}}

\startframedtext
\starttyping
\report [.1.] {.2.} {.3.} {.4.}
\stoptyping
\startargumenttable
\NC  1  \NC  ref,...  \NC  label(s) om naar te verwijzen \emph{(optioneel)}  \NR
\NC  2  \NC  tekst    \NC  kop, gebruikt als hoofdstuktitel                  \NR
\NC  3  \NC  tekst    \NC  datum, komt onder de kop te staan                 \NR
\NC  4  \NC  tekst    \NC  auteur, komt onder de kop te staan                \NR
\stopargumenttable
\stopframedtext

Zoals gezegd is de enige toegevoegde waarde van \tex{report} dat we de datum en de auteur aan een hoofdstuk kunnen meegeven. De volgende twee regels zijn dan ook geheel equivalent.

\starttyping
\report{Een hoodstuk zonder extra informatie}{}{}
\chapter{Een hoodstuk zonder extra informatie}
\stoptyping

\subsection{\tex{personalia}}

\startframedtext
\startsidebysidetext
Compacte vorm:
\starttyping
\personalia .1.
         \\ .2.
         \\ .3. \\
\stoptyping
\nextsidebysidetext
Uitgebreide vorm:
\starttyping
\startpersonalia
  .1.
\nextpersonalia
  .2.
\nextpersonalia
  .3.
\stoppersonalia
\stoptyping
\stopsidebysidetext
\startargumenttable
\NC  1  \NC  tekst  \NC  tekst in eerste kolom (bijv. pasfoto)       \NR
\NC  2  \NC  tekst  \NC  tekst in tweede kolom (bijv. adres)         \NR
\NC  3  \NC  tekst  \NC  tekst in derde kolom (bijv. telefoonnummer) \NR
\stopargumenttable
\stopframedtext

Voor de reader en het draaiboek is het handig om op een overzichtelijke manier de persoonlijke gegevens van alle reisdeelnemers te zetten. Hiervoor hebben we het commando \tex{personalia} toegevoegd.

Met dit commando kunnen we drie kolommen zetten met daarin bijvoorbeeld een pasfoto, het adres en het telefoonnummer van de deelnemer. Dit is een voorbeeld van zogenaamde \emph{paragrafen} (\emph{\en paragraphs}) in \CONTEXT\seedocumentation{12}{4.11}. We kunnen ze dan ook op twee manieren gebruiken: de compacte vorm en de uitgebreide vorm. De uitvoer is exact hetzelfde.

\startbuffer
\personalia \externalfigure[mariecurie][passphoto]
         \\ \startlines
            Marie Curie
            Heyendaalseweg 135
            6525 AJ Nijmegen
            \stoplines
         \\ (024) 36 52 079 \\

\startpersonalia
  \externalfigure[pierrecurie][passphoto]
\nextpersonalia
  \startlines
  Pierre Curie
  Heyendaalseweg 135
  6525 AJ Nijmegen
  \stoplines
\nextpersonalia
  (024) 36 52 079
\stoppersonalia
\stopbuffer
\typebuffer

De \tex{start/stoplines} zorgen er voor dat de regeleinden behouden blijven. Het tweede argument van \tex{externalfigure} is een voorgedefinieerde grootte. Zo hoef je er niet over na te denken hoe groot de afbeelding moet zijn en het scheelt tikwerk. Het resultaat is als volgt.

\getbuffer

\section{Plaatsen van figuren}

Voor het plaatsen van figuren gebruiken we het standaard \CONTEXT-commando \tex{placefigure}. Het staat uitgebreid beschreven in de \from[excursion] en de \from[manual], maar omdat het zo een belangrijk onderdeel is van een verslag, leggen we het hier nogmaals uit. Een ander handig \CONTEXT-commando is \tex{start/stopcombination}. Hiermee kunnen we figuren naast elkaar of onder elkaar zetten.

\subsection{\tex{placefigure}}

Het plaatsen van figuren in een technisch document is altijd lastig. De makers van \CONTEXT\ hebben hier extra aandacht aan besteed zodat \CONTEXT\ om kan gaan met grote hoeveelheden figuren. Met \tex{placefigure} kun je zowel figuren tussen de tekst plaatsen als naast de tekst of op aparte pagina's\seedocumentation{10}{12.2}. Het commando \tex{placemarginfigure} hebben we toegevoegd om figuren in de marge te plaatsen of om ze uit de marge te laten steken.

\startframedtext
\starttyping
\placefigure       [.1.] [.2.] {.3.} {.4.}
\placemarginfigure [.1.] [.2.] {.3.} {.4.}
\stoptyping
\startargumenttable
\NC  1   \NC   ..,..    \NC   plaatsingsopties \emph{(optioneel)}                         \NR
\NC  2   \NC   ref,...  \NC   label(s) om naar te verwijzen \emph{(optioneel)}            \NR
\NC  3   \NC   tekst    \NC   bijschrift, laat dit leeg als je geen bijschrift wilt       \NR
\NC  4   \NC   tekst    \NC   inhoud, meestal wil je hier \tex{externalfigure} gebruiken\footnote{Maar je kunt natuurlijk ook een tabel plaatsen en die samen nummeren met de figuren. Mocht je dat om een of andere onverklaarbare reden willen\dots}  \NR
\stopargumenttable
\stopframedtext

\tex{placemarginfigure} is niet zozeer een nieuw commando. Het is een kloon die we \CONTEXT\ hebben laten maken van de originele \tex{placefigure}. Hierbij hebben we een opmaak voorgedefinieerd, namelijk dat het figuur in de marge komt te staan. Het gebruik van dit commando is verder geheel analoog aan het origineel.

Een standaard figuur is een zogenaamde \emph{float}. Dat betekend dat, wanneer \CONTEXT\ geen plek kan vinden voor het figuur op de huidige pagina, hij op zoek gaat naar ruimte iets verderop in het document. Door een plaatsingsoptie mee te geven kunnen we dit uitstel proces beïnvloeden. De meest gebruikte opties zijn te vinden in \in{tabel}[tab:plaatsing]. Dit geldt trouwens ook voor andere floats, zoals tabellen.

\placetable[][tab:plaatsing]
  {Voorkeuren voor figuurplaatsing}
\starttabulate[|lT|l|]
\HL
\RC  Optie   \RC  Betekenis                   \NR
\HL
\NC  top     \NC  boven aan de pagina         \NR
\NC  bottom  \NC  onder aan de pagina         \NR
\NC  page    \NC  op een nieuwe, lege pagina  \NR
\NC  left    \NC  links van de tekst          \NR
\NC  right   \NC  rechts van de tekst         \NR
\NC  inner   \NC  tegen de binnen marge       \NR
\NC  outer   \NC  tegen de buiten marge       \NR
\HL
\stoptabulate

De laatste vier opties uit \in{tabel}[tab:plaatsing] creëren een \emph{zijfiguur}. Dat wil zeggen dat de tekst langs het figuur doorloopt in plaats van boven en onder. Bij de opties \type{left} en \type{right} loop de tekst respectievelijk rechts en links langs het figuur. Het is echter beter om \type{inner} en \type{outer} te gebruiken. Op oneven pagina's hebben ze hetzelfde effect als \type{left} en \type{right}, maar op even pagina's is de definitie precies anders om.
%Maar wanneer er meerdere afbeeldingen op dezelfde pagina staan, is het mooi om een van de twee tegen de binnenmarge te zetten. Dit doen we met de plaatsingsoptie \type{inner}. We kunnen ook gebruikmaken van \type{left} en \type{right} wanneer we de figuur echt altijd tegen de linker respectievelijk rechter marge willen zetten, ongeacht op wat voor pagina we zitten.

\startbuffer
\placefigure[inner]
  {Onze aarde}
  {\externalfigure[earth][width=0.2\textwidth]}
\stopbuffer
\getbuffer

We laten nu een paar voorbeelden zien van figuren. Waar mogelijk laten we de optionele haakjes van \tex{placefigure} weg. Let op dat wanneer je geen plaatsing wilt opgeven, maar wel een label je het eerste paar rechte haakjes \emph{wel} meegeeft! De aarde is gezet met:

\typebuffer

Een zijfiguur hoort bij de paragraaf die \emph{onder} hem staat. Wanneer het figuur te groot is en niet meer op de pagina past, wordt niet alleen het figuur naar de volgende pagina verplaatst. De bijbehorende paragraaf gaat mee! Ook als er nog wel plaats is voor de tekst. Hier zijn een paar oplossingen voor.

\startbuffer
\placemarginfigure
  {Onze zon}
  {\externalfigure[sun][width=\marginwidth]}
\stopbuffer

\startitemize[n]
\head Maak het figuur groter of kleiner

In plaats van \type{0.6\textwidth} is \type{0.5\textwidth} misschien voldoende en past het nog wel op de pagina. Wil je dit niet, dan kun je misschien een figuur dat daarboven staat groter maken, zodat de pagina meer opvult.

\getbuffer
\head Zet het figuur in de marge

Vooral handig bij kleine figuren. Let er op dat je de breedte van het figuur dan gelijk maakt aan de breedte van de marge. Hier is \tex{marginwidth} erg handig voor.

\typebuffer

Je hoeft margefiguren niet gelijk te maken aan de margebreedte. Als je ze groter maakt steken ze uit de marge de tekst in. Dit kan ook handig zijn bij grote figuren. Zie bijvoorbeeld \in{figuur}[fig:zonnestelsel] op \at{pagina}[fig:zonnestelsel].

\head Link het figuur aan een andere paragraaf

Met andere woorden: verplaats het figuur in het bronbestand een paragraaf omhoog of omlaag. Vaak hoort een figuur toch bij meerdere paragrafen.

\head Maak het figuur alleenstaand

Ofwel, gebruik geen plaatsingsoptie. Deze figuren worden vastgehouden. Pas wanneer er ruimte is op een pagina plaatst \CONTEXT\ ze. Dit kan dus pas op de volgende pagina zijn, maar de tekst loopt gewoon door!

\head Zet het figuur op een aparte pagina

Als een afbeelding erg groot is, dan kan het handig zijn hier een aparte pagina voor te reserveren. Alles wat je hoeft te doen is de plaatsingsoptie \type{page} mee te geven aan \tex{placefigure}.\footnote{Kan ook bij \tex{placemarginfigure}, maar is een beetje vreemd vind je niet?}
\stopitemize

\startbuffer
\placemarginfigure[][fig:zonnestelsel]
  {Ons zonnestelsel met zijn planeten Mercurius, Venus, Aarde, Mars, Jupiter, Saturnus, Uranus en Neptunus (Pluto staat er helaas net niet op)}
  {\externalfigure[solarsystem][width=\textwidth]}
\stopbuffer
\getbuffer

Plaatjes zetten blijft altijd een beetje prutsen. Het is een goed idee om, bij het maken van het verslag, de stukjes afzonderlijk te compileren. Dat scheelt een hoop compileertijd! Verder moet je in je achterhoofd houden dat jij niet alleen de verslaglegger bent maar ook de typograaf. Het hangt van jou af hoe de pagina's er uit komen te zien. Probeer een paar ideeën en kijk wat \emph{jij} mooi vindt!

\typebuffer

\subsection{\tex{start/stopcombination}}

\startframedtext
\starttyping
\startcombination [.1.]
  {.2.} {.3.}
  ...
\stopcombination
\stoptyping
\startargumenttable
\NC  1  \NC  n*m    \NC  een matrix van $n$ kolommen en $m$ rijen                       \NR
\NC  2  \NC  tekst  \NC  inhoud van eerste matrixelement, meestal \tex{externalfigure}  \NR
\NC  3  \NC  tekst  \NC  bijschrift van eerste matrixelement, mag leeg zijn             \NR
\NC  .  \NC         \NC  nog $n \cdot m - 1$ haakjesparen met inhoud en bijschrijft     \NR
\stopargumenttable
\stopframedtext

Soms wil je meerdere (kleine) figuren naast elkaar of onder elkaar zetten. Ook hier heeft \CONTEXT\ een oplossing voor. Dit kan met behulp van \emph{combinaties} (\emph{\en combinations})\seedocumentation{10}{12.3}. We laten meteen een voorbeeld zien.

\startbuffer
\placefigure
  {Grote planeten}
  \startcombination[4*1]
    {\externalfigure[jupiter][height=0.12\textheight]}  {Jupiter}
    {\externalfigure[saturn] [height=0.12\textheight]}  {Saturnus}
    {\externalfigure[uranus] [height=0.12\textheight]}  {Uranus}
    {\externalfigure[neptune][height=0.12\textheight]}  {Neptunus}
  \stopcombination
\stopbuffer
\typebuffer

\getbuffer

\startbuffer
\placemarginfigure[][fig:margecombi]
  {Kleine planeten}
  \startcombination[1*3]
    {\externalfigure[mercury][width=\marginwidth]}  {Mercurius}
    {\externalfigure[venus]  [width=\marginwidth]}  {Venus}
    {\externalfigure[mars]   [width=\marginwidth]}  {Mars}
  \stopcombination
\stopbuffer
\getbuffer

Het argument tussen blokhaakjes geeft het aantal kolommen en rijen aan van je \emph{matrix}. Tussen de \tex{start/stopcombination} geef je voor elk element uit de matrix in een haakjespaar het figuur dat je wilt plaatsen en het bijschrift van dat figuur. Dat is alles. Alle soorten matrices zijn mogelijk. Combinaties kun je, net als gewone figuren, zowel in de tekst zetten, naast de tekst, op aparte pagina's in de marge (zie bijvoorbeeld \in{figuur}[fig:margecombi]), et cetera. De mogelijkheden zijn eindeloos!

\typebuffer

\todo{broad figures}

\todo{handmatig pagina invoegen}

\todo{\tex{blank}}

\todo{fontopmaak \tex{bf} \tex{it} etc}

\stopcomponent

% vim: ft=context spell spl=nl cole=1
