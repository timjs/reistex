\startcomponent inleiding
\product        handleiding
\environment    handleidinglayout

\chapter[chp:inleiding]{Inleiding}

Stukjes voor de reader en het reisverslag schrijven we in de opmaaktaal \CONTEXT. Dit om de cie een hoop werk te besparen, en om iedereen met \TEX\ om te leren gaan. Als je nog niet goed met \TEX\ kunt werken, vrees niet! In dit document staat alles wat je nodig hebt om een goed \TEX-bestand af te leveren aan de cie en om al die stukjes samen te voegen tot een mooi document.

Je kunt hier zowel informatie vinden voor de reisdeelnemer als voor de verslaglegger. Als reisdeelnemer vind je in \in{hoofdstuk}[chp:stukje] allerlei commando's en stijltips om een mooi\footnote{En bovenal gestructureerd.} stukje aan te leveren bij de cie. In \in{hoofdstuk}[chp:compileren] staat hoe je dit stukje vervolgens kunt omzetten tot een mooi \PDF'je. Ook vind je daar enkele tips voor de commandoregel, die van pas komen bij het gebruiken van \TEX.

\startnotmode[deelnemer]
De twee hoofdstukken daarna zijn vooral bedoeld voor de verslaglegger.\footnote{Door sommigen ook wel \TEX-slet genoemd, maar laten we dat woord hier maar niet gebruiken\dots} In \in{hoofdstuk}[chp:verslag] staat uitgebreid beschreven hoe hij/zij de stukjes van de deelnemers samen kan voegen tot een mooi verslag. Daarnaast kan de verslaglegger in \in{hoofdstuk}[chp:opmaak] enkele tips en trucs vinden om het geheel er nog stijlvoller en gelikter uit te laten zien.
\stopnotmode

Aan het eind van deze handleiding vind je nog een paar appendices. In \in{appendix}[app:symbolen] staat een lange lijst met wiskundige symbolen. Handig voor als je veel formules wilt zetten. Voor \LATEX'ers komt \in{appendix}[app:commandos] van pas. Daar staan veelgebruikte commando's in \LATEX\ en hun equivalenten in \CONTEXT. Tot slot vind je in \in{appendix}[app:bibtex] informatie over \BIBTEX. Het bevat een overzicht van alle documentsoorten, wanneer je ze gebruikt en welke velden je moet invullen.

Alle stukjes en ook dit document zijn gebaseerd op \type{travellayout.tex}. Hierin staat de opmaak van het reisverslag, de reader en de stukjes gedefinieerd. Ook bevat het enkele toevoegingen die het leven van een reisdeelnemer en verslaglegger leuker en makkelijker maken. Zonder \type{travellayout.tex} zal je document de standaard (saaie) \CONTEXT\ opmaak hebben. Zorg er dus voor dat je dit bestand altijd bij de hand hebt.\footnote{Wil je weten hoe we deze opmaak voor elkaar krijgen? Neem dan een kijkje in \type{travellayout.tex}.}

\subject{Documentatie}

\CONTEXT\ is een uitgebreid macropakket dat, net als \LATEX, gebouwd is rond \TEX. Het grootste verschil met \LATEX\ is dat \CONTEXT\ je veel meer vrijheid geeft in het definiëren van je eigen opmaak. Daarnaast is veel functionaliteit waar je bij \LATEX\ pakketten nodig hebt standaard ingebouwd.

Het aantal commando's en opties in \CONTEXT\ is enorm. We bespreken hier alleen de basis. Er zijn twee goede handleidingen op internet te vinden van de makers van \CONTEXT. De eerste is \emph{\CONTEXT\ an \from[excursion]} die elk onderdeel van een \CONTEXT-document kort bespreekt. De tweede is \emph{\CONTEXT\ the \from[manual]}.\footnote{Deze wordt momenteel herzien.} Dit is het meest uitgebreide document dat je over \CONTEXT\ kunt vinden en hier staat heel veel over het systeem beschreven.\footnote{Maar toch nog niet alles\dots} Daarnaast is er nog de \from[wiki] (\url[wiki]) waar heel veel extra informatie op te vinden is.

Af en toe zul je in de marge verwijzingen zien staan naar de \from[excursion] en de \from[manual] zoals hiernaast\seedocumentation{1}{1.2}. De nummers verwijzen naar het hoofdstuk of de paragraaf waarin je meer informatie kunt vinden over het huidige onderwerp. Op deze manier kun je eenvoudig de details vinden in de officiële \CONTEXT\ documentatie. Helaas kunnen we geen directe links maken naar de desbetreffende paragrafen, maar je kunt wel met één klik de documenten openen. Die zijn zelf interactief, dus je kunt daarin doorklikken naar de juiste sectie.

%De links in dit document zijn ook zo gemaakt dat je er op kunt klikken. 

Van bovenstaande documenten is ook een Nederlandstalige versie beschikbaar. Maar daarin worden ook de Nederlandstalige commando's gebruikt.\footnote{\CONTEXT\ is beschikbaar in verschillende talen.} Omdat we zowel teksten in het Engels als in het Nederlands gaan schrijven, gebruiken we in deze handleiding alleen de Engelstalige commando's.

Ik hoop dat je hier alle informatie kunt vinden die je nodig hebt bij het schrijven van je stukje of het opmaken van het reisverslag of de reader. Ontbreekt er iets of heb je nog andere op- of aanmerkingen? Schroom niet om de cie te mailen! Zij zullen er voor zorgen dat jouw goede tips verwerkt worden.

\startalignment[flushright]
\startlines
Tim Steenvoorden
Nijmegen, april 2011
\stoplines
\stopalignment

\stopcomponent

% vim: ft=context spell spl=nl cole=1
