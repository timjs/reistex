\startcomponent ioffe_afternoon
\environment    travellayout

\startchapter[title=Ioffe Physical Technical Institute\\ (the afternoon),
              label=fys:ioffe_aft]
             [date={\date[d=13,m=10,y=2009]},
              author=Maarten van de Griend \& Martijn Jongen]

\abbreviation{FIZTEKH}{Ioffe Physical Technical Institute}
\abbreviation{PTHS}   {Physical Technical High School}
\abbreviation{JET}    {Joint European Torus}
\abbreviation{ITER}   {International Thermonuclear Experimental Reactor}
\abbreviation{UK}     {United Kingdom}
\abbreviation{START}  {Small Tight Aspect Ratio Tokamak}
\abbreviation{MAST}   {Mega Amp Spherical Tokamak}
\logo        [GLOBUSM]{\cap{globus-m}}

\section{History of the Ioffe institute}

\placemarginfigure
  {Logo of the Ioffe institute}
  {\externalfigure[ioffe_logo][margin]}

The history of the \infull{FIZTEKH} (or \FIZTEKH, as it is called informally) dates back all the way to 1918. At that time it was just a department of an institute called the State Institute of X-Rays and Radiology. It was called the \quote{Physical Technical Department} and it was founded by A.F. Ioffe, who also headed it for over thirty years.

In November 1921 the State Institute of X-Rays and Radiology split up, and the physical-technical department became an institute of its own: the State Physical Technical X-Ray Institute. By that time the staff consisted of only a few dozen people. In 1923 the institute moved into it's own building. This building is currently still the main building of the Ioffe institute. After the death of Ioffe in 1960 the institute was named after him.

Many prominent scientists, including several Nobel Prize laureates are associated with the Ioffe institute. Since this article is supposed to have an explicit physical emphasis, we shall mention them only briefly for completeness. A few anecdotes are also included.

\startitemize
\item A.F. Ioffe ({\ru Abram Fedorovits Ioffe})
\item Nicolai Semenov ({\ru Nikolaï Semenov}, Nobel Prize in Chemistry 1956)
\item Pjotr Leonidovich Kapitsa ({\ru Pëtr Leonidovich Kapitsa}, Nobel Prize in Physics 1978)
\item Lev Davidovich Landau ({\ru Lev Davidovich Landau}, Nobel Prize in Physics 1962)
\item G.A. Gamow ({\ru G. Gamov})
\item Igor Tamm ({\ru Igorp Tamm}, Nobel Prize in Physics 1958)
\item Zhores Alferov ({\ru Zhores Alferov}, Nobel Prize in Physics 2000)
\stopitemize

\placefigure[inner]
  {The Ioffe institute main building}
  {\externalfigure[ioffe][large]}

Ioffe was not an exceptional scientist but a great organizer of science. He helped with the founding of many laboratories and institutes. In this way he contributed a lot to science. He is so famous that there are Soviet Union postal stamps featuring his portret and a crater on the moon was even named after him.

Kapitsa lost his entire family in the Spanish Flu epidemy of 1920. Because of this Ioffe took him on a journey through Europe, where he eventually stayed with Rutherford at the famous Cavendish laboatory. Initially there was no vacancy, but Ioffe persuaded Rutherford to create one. He asked Rutherford: \quote{What is the average error in your measurements?} upon which Rutherford replied \quote{About 10 percent.} \quote{How many people work in your laboratory?} \quote{31.} \quote{Then,} Ioffe said, \quote{you can create a free position!}

Lev Landau got arrested for making an anti-communist leaflet. He was saved from execution by a letter from Kapitsa to Stalin himself, so Landau was instead sent to prison. The story goes that Landau's mother sent him money while he was in jail, but he never received it. The guards had stolen it, because they were almost certain he would never leave prison.

\section{Physical Technical School}

When we visited the Ioffe institute, we were promised a visit to the physical technical high school in the afternoon. We went towards the school as planned by walking from the Ioffe institute through a beautiful forest. When we finally arrived at the school's entrance, we were about to find out that there was some problem with our security clearance so the entire trip unfortunately got cancelled. We will tell you about the school and it's history, despite us not entering the buildings. The \infull{PTHS} (\PTHS) is a school for pupils in the last four years of secondary school. This is a very special school because it crams students for work as an independant researcher. It is also the only school supported by the Russian Academy of Sciences. The main courses given are physics, mathematics, english and computer sciences. It has an extensive program which includes six days a week learning, but also included in these six days, four hours a week of intense physical training. The system stimulates competitiveness between the students, because grants are awarded for the best students. About half of the students continue their studies at the Ioffe institute, while the rest engage their activities elsewhere. Studying at the \PTHS\ is completely free of charge, admission takes place through tests. A quarter of the students later end up at the Ioffe institute as independent researchers.

\section{Tokamak}

 The tokamak is a Russian invention. The word Tokamak is an acronym for \cap{to}roidal'naya \cap{ka}mera \cap{ma}gnitnymi \cap{k}atushkami, which means: toroidal chamber with magnetic coils. For instance, this machine is very promising for the production of nuclear energy. Some scientists believe that in a world with an increased need for energy, nuclear fusion energy may play a key role in solving this problem. The resources needed are in abundance on our planet and a reactor does not produce any greenhouse gases or any nuclear waste, nor is there a possibility that a nuclear meltdown occurs, like in a nuclear fission reactor. The main problem with the tokamak device is to build the device large enough to get a high enough yield. Scientists are yet to build a machine with a higher output energy than the input power. We shall explain now how a tokamak operates to produce energy.

A tokamak is a toroidal space in which a plasma of deuterium and tritium is ejected which is kept in place through means of magnetic fields to prevent unwanted interactions with the surroundings. It is then heated up to very high temperatures and pressures in order to let nuclear fusion take place. Deuterium fuses with tritium to form helium and a loose neutron, so it is not affected by the magnetic field. This happens when they get within \quantity{1}{fm} of one another, hence the high required temperature. The neutron is given a lot of kinetic energy by the mass defect of the reaction, so it traverses to the chamber walls at high speed. When the neutron hits the chamber walls, it is supposed to hit a lithium ion, forming helium and tritium. Along with this fusion reaction, the energy is passed on in the form of heat. This in turn is passed on to water in a pressurized pipe, which makes other water boil under a lower pressure. The steam is used to power a generator and finally we have our electricity.

Back in the reaction chamber, the helium nucleus is also given some energy, which is used to keep the plasma at the high temperature. Since the plasma has no interactions with the wall, there is also no heat transfer through conductance. This leaves radiation to be the only way to transfer heat. But at order of $10^7$~degrees, this radiance is extremely high, since $L=\sigma AT^4$. It is therefore more efficient, to keep the area of the outside of the plasma as small as possible, relative to the amount of plasma. This is why bigger reaction chambers are better than small ones.

Now there is a problem getting the plasma initially at a high enough temperature. There are several heating methods available. First, there is ohmic heating, which is basically sending an electrical current through the plasma, and by the resistance of the plasma, energy dissipates into heat. Another way to heat the plasma is simply to compress it. This is done by increasing the magnetic field. This requires stronger currents, and it is therefore that such high magnetic fields are used. Other methods of heating the plasma are described below.

\section{Tokamak types}

There are several types of fusion reactors possible for successfully producing electrical energy. For example the \emph{gas dynamic trap} we saw at the Budker Institute, or \emph{laser inertial confinement} in which a small pellet of fusion material is shot with an extremely intense laser pulse. But the tokamak is the most advanced fusion technique so far.

Examples of tokamak fusion facilities are \JET\footnote{\infull{JET}, operating at the Culham Science Center in the \UK.} and \ITER\footnote{\infull{ITER}, or \quote{the way} in Latin, currently under construction near Cadarache in southern France.}, both among the largest fusion projects in the world. Countries all over the world contribute to these experiments and it is hoped they will demonstrate the possibility of commercial fusion energy production.

According to the researchers at the Ioffe institute, there might be a better way to construct a tokamak than is currently being done at \ITER. A different shape of tokamak might function better. This difference in shape is defined by the \emph{aspect ratio}: the ratio between the major axis and the minor axis of the toroidal (doughnut-shaped) plasma. In simple language this means the ratio between the height and the width of the plasma. Conventional tokamaks like \ITER\ have a plasma that is much wider than it is high. An alternative is to make the plasma equally high as it is wide, creating a spherical shape.

\placefigure[inner]
  {Plasma in \JET, a conventional tokamak}
  {\externalfigure[JETplasma][large]}

An earlier experiment at the Culham Science Center called the \infull{START} (\START) showed that the spherical tokamak design is very promising. The design of \START\ was much smaller, simpler and cheaper than that of a conventional tokamak, but despite this it achieved some excellent results. For example the magnetic field needed to stabilize the plasma was a factor 10 lower than usual. It also set the world record for plasma containment efficiency. To be precise, it reached the highest $\beta$-factor, a measure commonly used in plasma physics, which is defined as the ratio of plasma pressure to magnetic field pressure:
\startformula
\beta=\frac{p}{p_{\text{mag}}}=\frac{nk_B T}{B^2/2\mu_0}.
\stopformula
The \START\ experiment was followed by the \infull{MAST} (\MAST), which continues to explore the possibilities of the spherical tokamak design.

The aspect ratio for \JET\ is about 3. The same aspect ratio is planned for \ITER. \START\ and \MAST\ have an aspect ratio of about 1.3.

\section{The tokamak at the Ioffe institute}

\placefigure[outer]
  {The \GLOBUSM\ tokamak at the Ioffe institute}
  {\externalfigure[globus][small]}

The tokamak at Ioffe, called \GLOBUSM, is also meant to explore the possible advantages of the spherical tokamak over the conventional model. It is used to study high temperature plasmas with specific application to fusion. Only the plasma is researched though, so no actual fusion takes place. Recent research topics include heating methods, diagnostics, and plasma-wall interaction. These terms will be explained below.

The main plasma heating mechanism in \GLOBUSM\ is \emph{neutral beam injection}. This means a high-energy beam (\quantity{20-29}{keV}) of hydrogen or deuterium atoms is shot into the plasma. This does not contaminate the plasma, because the plasma itself consists of these elements. Due to collisions with the plasma inside the tokamak the atoms become ionized and are consequently captured in the magnetic field where they carry over their kinetic energy to the rest of the plasma.

\placefigure[outer]
  {A schematic overview of the \GLOBUSM\ experimental setup. NBI: Neutral Beam Injector, CX NPAs: Charge Exchange Neutral Particle Analysers, TS: Thomson Scattering. ICRH: Ion Cyclotron Resonance Heating. A description of the individual parts can be found in the text.}
  {\externalfigure[globus_scheme][large]}

Since normal accelerators only work for charged particles, one might wonder how the atom beam is created in the first place. To achieve this, the atoms are first ionized. The ions are then accelerated into a beam which is then led through a gas target, where some of the ions are neutralized by interaction with the gas particles. The remaining ions are then deflected out of the beam by a magnetic field.

The plasma is also heated using \emph{ion cyclotron resonance heating}. This technique uses electromagnetic radiation at the cyclotron resonance frequency to heat ions. The \emph{cyclotron frequency} is the frequency at which charged particles orbit in a homogenous magnetic field and is given by
  \startformula
  f_c=\frac{Bq}{\pi m}.
  \stopformula

The cyclotron resonance frequency depends on the magnetic field strength, which in turn varies over position in the tokamak. This property can be used to study energy transport by locally heating the plasma. In \GLOBUSM\ typical frequencies of 7 to \quantity{9}{MHz} are used. The beam power is 0.1 to \quantity{0.3}{MW}.

Diagnostics are done using Thomson scattering. This is the scattering of electromagnetic radiation off charged particles -- in this case the electrons in the plasma. The radiation is scattered in all directions\footnote{the intensity of the scattered radiation depends on the scattering angle, but we will not go into this in detail, since it is not relevant here.}. The intensity of the scattered radiation depends linearly on the density and flux of the charged particles. When we define $\epsilon$ as the amount of energy scattered by a volume element of charged particles $\total V$ into an angle $\total\Omega$ in a time $\total t$ and a wavelength between $\lambda$ and $\lambda+\total\lambda$, the following relation holds:
  \startformula
  \epsilon=\frac{\pi\sigma}{2}In.
  \stopformula
Here $n$ is the density of charged particles, $I$ is the incoming radiation flux and $\sigma$ is the Thomson differential cross section defined as
  \startformula
  \sigma={\left(\frac{q^{2}}{4\pi\epsilon_{0}mc^2}\right)}^2.
  \stopformula

Since $\sigma$ is a constant for a given type of particle (electrons for instance), the electron density can be calculated from the intensity of the scattered radiation.

Because the plasma is very hot, the electrons move around rapidly in random directions. This causes Doppler-shifting of the emitted radiation. When a single laser frequency is used, the single line in the spectrum will broaden. This effect can be used to calculate the temperature of the electrons.

The walls of \GLOBUSM\ experience heavy exposure to plasma: up to several \unit{MW/m}. Therefore the inner wall is covered with protective tiles. These tiles are given special Boron-based layers to improve the working of the tokamak. After the tiles have been exposed to the plasma for a long time, their chemical properties are investigated.

Normally the machine runs all day long, creating a shot (discharge) every \quantity{5-6}{minutes}. This costs about half a million dollars of electricity every year.

\section{Tokamak comparison}

As described above, the aspect ratio of \GLOBUSM\ is much smaller than that of \JET\ and \ITER. There are many parameters with which you can compare different tokamaks. A bit of information about them can be found in \in{tabel}[info].

\placetable[][info]
  {A comparision of different tokamaks}
\starttabulate[|l|l|l|l|]
\FL
\RC                \RC \GLOBUSM            \RC \ITER              \RC \JET               \NR
\ML
\NC Major radius   \NC \quantity{0.36}{m}  \NC \quantity{6.2}{m}  \NC \quantity{4.2}{m}  \NR
\NC Minor radius   \NC \quantity{0.24}{m}  \NC \quantity{2.0}{m}  \NC \quantity{2.5}{m}  \NR
\NC Aspect ratio   \NC 1.5                 \NC 3.1                \NC 1.7                \NR
\NC Magnetic field \NC \quantity{<0.62}{T} \NC \quantity{5.3}{T}  \NC \quantity{<3.5}{T} \NR
\NC Current        \NC \quantity{<0.5}{kA} \NC \quantity{15}{kA}  \NC \quantity{<5}{kA}  \NR
\NC Power          \NC \quantity{0}{MW}    \NC \quantity{500}{MW} \NC \quantity{16}{MW}  \NR
\LL
\stoptabulate

The reason that the power for the \GLOBUSM\ is denoted as \quantity{0}{MW} is that no nuclear fusion takes place here. The device is merely for plasma properties studies. That is also the reason for its small scale. As explaned above, the aspect ratio is also low. From all the stats can be seen, that \ITER\ will become larger and more impressive.

% Onderstaande omdat deze publicaties niet geciteerd zijn, maar wel in de lijst moeten --TS
\nocite[globus,ioffe,iter,plasma,fusion,thomson,mast,jet,tokamak,pths,nbi]

\stopchapter

\stopcomponent
