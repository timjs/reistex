\startcomponent before
\environment    travellayout

\startchapter[title=Lecture from Prof. Katsnelson]
             [date={\date[d=18,m=9,y=2009]},
              author=Sjoerd Verhagen]

\section{Preparational talk}

Such a long journey far abroad can't do without thorough preparation. Besides polishing up the participants knowledge about physics, a good background study gives insight in the general culture of the country being visited. And last but not least: it fosters enthusiasm for the great travels to come. Professor Katsnelson was so kind to give us a preparational lecture in which he mixed Russian culture, history and physics.

\section{Professor Katsnelson}

Born in 1957, Prof. Katsnelson wasted no time working his way into the world of physics. At the age of only fifteen, he finished a high school of advanced physics and mathematics. Five years later, a master of science earned in the Ural State University of Sverdlovsk opened the door to a \cap{phd} in Solid State Physics in the same city. In 1992 he became Professor of Solid State and Mathematical Physics at the Ural State University. Since 2004 he holds the post of Professor in theoretical physics in the Radboud Universty of Nijmegen.

\section{Physics in Russia}

Substantial scientific work in Russia started in the 18\high{th} century. In seventeen14, Peter the Great, ruler of Russia, decided that all of the children of the nobility should have some early education, especially in the areas of sciences. He issued a decree which dictated that all Russian children of the nobility, government clerks, and even lesser ranked officials, must learn basic mathematics and geometry, between the ages of 10 and~15 years. In this way, the basis for a higher educated society was laid. A few decades later, Mikhail Lomonosov, an outstanding Russian scientist, founded the Moscow State University in 1755. Heroes from this period include Lobochevsky, who pioneered in non-euclidian geometry and Lomonosov, who stated the ideo of conservation of matter, improved the design of the reflecting telescope and gave a theoretical prediction of the existence of Antarctica.

In the 19\high{th} century, the famous Mendeleev invented the periodic table. In the 20\high{th} century, however, things got really started. A lot of Russians were working abroad at that time. The government stimulated this practice to boost the level of their scientists. Ehrenfest for example moved from Saint Petersburg to work in Leiden. Lev Landau worked in Copenhagen with Niels Bohr before he discovered diamagnetism. Pyotr Lebedev, who did his doctoral degree in Strasbourg, performed the difficult measurement of the momentum of light via the pressure it applies. A truely great organizer of science from that period is Abram Ioffe, who was at that time studying under supervision of Wilhelm Röntgen in Germany. When he got back to Russia, he initiated semiconductor research in which he played an important role as organiser of science. After Wilson explained matals with conduction band structures, this research became very important. But the best example is Pyotr Kapitsa, the greatest Russian experimentalist. With help from Ioffe, Kapitsa got a position in Rutherfords laboratory where he worked on low temperatures and magnetic fields. Kapitsa was half engineer, half physicist. This combination made him a very succesful experimentalist. When Kapitsa was old he said the healthy ratio between theoreticians and experimentalists is 1 to 10. In the \cap{ussr} it was 1 to 1\periods\ In 1920 he succeeded in producing liquid helium in an efficient way. 

The fact that Russian scientist were encouraged to attain experience and knowledge abroad did not mean they were free to remain in a country of their choice. Those scientists who decided not to come back to the motherland could not expect much understanding. After George Gamow, Nobel laurate for cracking the $\alpha$ decay problem, did not return from abroad, he was demonised.

Although the Russian universities had troubles with anti-scientific attitudes after the revolution of 19seventeen\footnote{Special Relativity and QM were not allowed to be taught at one point}, the A-bomb development put physics back on the national agenda. This did however overdevelop certain branches of physics resulting in an unbalanced total.

Taking into consideration the good quality of Russian research and the international respect it enjoys, Prof. Katsnelson stressed that one should not allow these great men of the past eclips the current researchers working in Russia. 

\section{Crystals}

Everything solid is made out of crystals. The only possible exception is glass, but whether this type of materials should be classified solid is disputed. There are of course many different crystal structures possible. Though lattice distance can differ, essentially differences come from the symmetries lattices possess. Exactly 230 of these so called space groups can be found in Euclidean 3D. A space group is the combination of a crystallographic point group with a Bravais lattice.

A crystallographic point group is the group of symmetry operations which when applied on a lattice bring every lattice point to the spot another occupied before. The symmetry operations are:

\startitemize
\head Rotation

The rotation of a crystal about an axis, shown at \at{page}[symmetrie], figure a.

\head Reflection

Translating all points to the opposite side of a plane, shown at \at{page}[symmetrie], figure b.

\head Glide planes

A reflection in a plane together with a translation in this plane, called a glide reflection along a glide line, shown at \at{page}[symmetrie], figure c.

\head Rotoinversion

First rotating all points about an axis and subsequently reflecting in the plane perpendicular to the axis, shown at \at{page}[symmetrie], figure d.
\stopitemize

\placefigure[outer][symmetrie]
  {The various possible symmetry operations on two dimensional lattices.}
  {\externalfigure[symmetrie][narrow]}

This leads to 32 point groups.

A Bravais lattice gives the translational symmetries of a crystal. It is defined by three vectors. With a lattice point in the origin, one can find a lattice point on each integer linear combination of the vectors. Such a lattice point may comprise multiple atoms. There are 14 of these Bravais lattices.

Not every combination of the 14 Bravais lattices with the 32 point groups constitutes a unique space group. Since a detailed description about the point group and Bravais lattice counting would be out of the scope of Prof. Katsnelson his talk, I will instead reveal the seventeen possible space groups in 2D. These groups give all possibilities for texture art on surfaces and have all been in use since the ancient times. Still, all these patterns are in extensive use throughout the muslim world \cite[Grunbaum:1990].

The first proof that there are only seventeen periodical lattice symmetry groups in two dimensions was given by the Russian E.S. Fedorov in 1891. Because it was written in Russian, Fedorov's important result was not known until G. Pólya and P. Niggli rediscovered it many years later.

\placefigure[inner][bewijs]
  {Crystallographic restriction theorem proof. $n$-fold rotational centres are shown of $n$-value higher than 6.}
  {\externalfigure[bewijs5fold][narrow]}

An important reduction in the possibilities to apply a pattern to a wall is given by the crystallographic restriction theorem. It states that rotational symmetry centres on 2 dimensional lattices only exist in 2-, 3-, 4- or 6-folds. It can easily be proven:
Assume a periodical lattice contains a $n$-fold rotation symmetry centre. Because of translational symmetry posessed by all lattices, there is a whole set of these rotational symmetry centres. Let us take one of these centers: $A$. There are one or more other $n$-fold rotation symmetry centres closest to $A$. Let us take one: $B$. If we rotate 2$\pi/n$ around the rotation centre $B$, this moves $A$ to a new position: $C$. Because of the rotational symmetry: $C$ is a $n$-fold rotation center. Again rotating 2$\pi/n$ but this time around $C$ moves $B$ to a new position: $D$, which is also a $n$-fold center. Basic geometry reveals that if $n$ is bigger than 6, $C$ would be closer to $A$ than $B$. This contradicts the fact that $B$ was (one of the points) closest to $A$! This contradiction is illustrated in the figure at the bottom of \at{page}[bewijs]. If $n$ equals 5, $D$ would again be closer to $A$ than $B$, giving rise to the same contradiction. The value of 6 for $n$ would be possible as it brings $B$ over in $A$ in the second rotation. Thus, values greater than 6 or five are not allowed for $n$.

To prove there are only seventeen possible 2D lattice symmetry groups one can employ Euler Characteristics. Euler Characteristics in 3D states the number of corners minus the number of edges plus the number of faces always equals 2. This brought down to two dimensions demands the features of a 2D lattice group to sum up to 2 via the following value assignment:
  \startitemize
  \item A center of $n$-fold rotation counts as $(n-1)/n$.
  \item A center of $n$-fold rotation with one or more crossing mirrors counts as $(n-1)/2n$.
  \item A (glide) reflection counts as 1.
  \item A total absence of symmetry counts as 2.
  \stopitemize
Arithmetic of the possible combinations leading to the desired number gives the seventeen symmetry groups.

Determination of your parents old rug symmetry is easy with the following guide. Just answer the questions in the figure below and move along. 

\placefigure[page]
  {Caption which was not there before}
  {\externalfigure[wall-flow][page]}

Examples of these seventeen groups together with their symmetry visualised can be seen in the figures at \at{page}[combination]. 

\placefigure[][combination]
  {Different symmetry groups}
  \startcombination[2*2]
    {\externalfigure[p4]  [medium]} {The p4 symmetry group.}
    {\externalfigure[p31m][medium]} {The p31m symmetry group.}
    {\externalfigure[pm]  [medium]} {The pm symmetry group.}
    {\externalfigure[pmg] [medium]} {The pmg symmetry group.}
  \stopcombination

On the left, an esthetic wall pattern is shown, on the right the symmetry properties are schemetically displayed.

\starttabulate[|l|l|]
\NC $\diamond$       \NC denotes for a two fold rotation center.   \NR
\NC $\bigtriangleup$ \NC denotes for a three fold rotation center. \NR
\NC $\square$        \NC denotes a four fold rotation center.      \NR
\NC Hexagon          \NC denotes a six fold rotation center.       \NR
\NC Line             \NC denotes a reflection line.                \NR
\NC Dotted line      \NC denotes a glide reflection line.          \NR
\stoptabulate

\page[bigpreference]
\section{X-rays}

\placefigure[right][laue]
  {The Von Laue diffraction experiment.}
  {\externalfigure[lauesetup][medium]}
 
The idea of atoms as the smallest building blocks of matter has been around since ancient Greek times. However, in 1910 the First experimental proof of this hypothesis has been found. It was Max von Laue who found out that X-raybibis could be diffracted on many solids to produce interference patterns. The nature of X-rays was not clearly understood back then, but the analogy with visible light through a diffracting grating was compelling. The combined theory of X-rays being light and solids consisting of atom crystals was subsequently printed in the textbooks. For this, Von Laue was awarded the Nobel prize in 1914.

\placefigure[outer][lauediffraction]
  {Laue diffraction pattern.}
  {\externalfigure[lauediffraction][small]}

The experiment set up by Von Laue now accounts for one of the most popular techniques for material analysis. A tube emits X-rays through a collimator and often a monochromator (not depited) to produce a parallel monochromatic beam which strikes the material which is to be examined. This setup is shown at \at{page}[laue]. If a set of parallel planes in the crystal structure are ordened in such a way that reflection from one plane interferes constructively with the next, the reflected beam will gather much intensity \cite[Fewster:2003]. This condition of X-rays with wavelength $\lambda$ diffracting constructively on a set of d distanced planes upon incidence at angle $\theta$ is called Braggs law and states:
  \startformula
  2d\sin\theta=n\lambda,
  \stopformula
where $n$ is a integer.

Such refracted beams can be detected by a photographic film or a movable X-ray detector. With the symmetries most crystal possess, multiple sets of planes can diffract simultaneously, giving both beautiful patterns and much information about the structure and properties. An example is given in the figure at \at{page}[lauediffraction].

\section{Mandelstam/Raman scattering}

In 1914, the Russian doctor Mandelstam returned from Germany to the Moscow State University in his homeland Russia. His field was optics and radio physics. Independent from C.V.~Raman, the \cap{us} based Indian physicist, he discovered the inelastic scattering of light a week before Ramans eyes first witnessed the effect. But Mandelstam, being a theoretician, spent too much time on theoretical considerations on the new phenomenon to be the first to publish. As a result, the effect now beares Raman's name. Also, in 1930, Raman received the Nobel prize of physics for this effect which greatly influences chemistry being a significant tool for analyzing the composition of liquids, gases and solids. 

\placefigure[outer][mandelstam]
  {Mandelstam Raman scattering.}
  {\externalfigure[raman][medium]}

Raman scattering is the inelastic scattering of light off the material or molecule to be analyzed. Quantum mechanics only allows certain energy states in materials or molecules. Because of this, inelastic scattering can only occur when incident and scattered photon energy difference equals the energy difference between two states. Each energy state therefore contributes a particular shift away from the incident wavelength to the scattered beam. Analyzing this scattered beam reveals the energy levels present \cite[Brehm:1989]. The concept is schematically illustrated in the figure at \at{page}[mandelstam].

Russia is a country rich in physics and high class physicists. Particularly the field of condensed matter physics is very popular. This beautiful field combines deep theoretical considerations with multiple effective experimental techniques. It is a great working area where experiments and theory stand close together.

\stopchapter

\stopcomponent
