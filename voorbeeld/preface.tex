\startcomponent preface

\report{Preface}{}{Andrei Kiriliouk}

A trip to Russia sounds in many aspects rather exotic – and in several cases, this is also justified. For example, where else would you volunteer to spend 50 hours in the train instead of just 3 hours in a plane? Or, when you just need to switch between two terminals of an airport (Pulkovo I and II), which is practically a walking distance, where else would you have to take a bus to the city, get out of it in the middle of nowhere, drag your bags through lots of stairs to cross the road, and take another bus back to the airport? 

However, the scientific part of the tour brings you back to a more normal life of a physicist. True, some of the laboratories we visited are crying for investments, and the buildings for a layer of fresh paint – but this is not unusual for many places in Europe either. Okay, sometimes people have difficulties to tell you about their research in a comprehensible language – but you see this in many other places, particularly if you travel in the southern European countries. Nevertheless, the physics by itself they were talking about, was not only understandable, but also in many occasions quite intriguing. I hope that this internationality of physics has been noticed by all the students. 

I was duly impressed by the scientific programme that was put together by the organizers, which was covering most, if not all, aspects of physics not only from the scientific, but also from the historical perspective. Thus, our visit to the Pulkovo observatory taught us the history of developments of Russian research in astronomy, from the beginning of the 19th century and into our days. The visit to Chernogolovka challenged our ability to understand rather involved scientific presentations on such topics as quantum dots and qubits or string theories. Our day at Moscow State University nicely combined these aspects to the point that also the high heels of the female security guards didn't remain unnoticed. And in all these institutions our guides were deeply impressed by the activity of our students (what they actually told me in Russian after each visit), their wish to understand what's going on, and ability to ask questions. Well done, guys!

Russia is the country with very deep cultural traditions, and I was very happy to see that this aspect of our tours was also taken quite seriously, with so many hours city walking, musea visits, and not to forget the obligatory visit to the \quote{Swan lake} ballet in Novosibirsk. 

The technical organization of this trip deserves a separate mentioning. I didn't even dare to ask how much time was invested just in that little book that each cie member always had with him, and that contained all our movements down to a minute, with bus and metro routes, change stations, alternatives, etc. And the fact that all organizers seriously studied Russian for the whole year, to be able to better communicate during the trip!

All this naturally resulted in a very successful study tour; I hope that the students find it not only interesting, but also very useful in giving them a certain perspective on physics as ultimately international area of science.

\stopcomponent
