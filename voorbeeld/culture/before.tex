\startcomponent before
\environment    travellayout

\report{Russia versus Europe}
       {\date[d=18,m=9,y=2009]} {Inka Locht}

\section{The First lecture}

\placemarginfigure
  {Dr.\,J.H.C. Kern}
  {\externalfigure[henkkern][width=\marginwidth]}

Henk Kern, lecturer of Russian history at the University of Leiden, told us about the history of Russia. His main question was: how is Russia marked by the heritage of its history? For this he identified eight differences between the Russian nation and most European nations, which are described in the following paragraphs.

\subsection{Autocracy instead of constitutionalism}

The kings of Kiev ruled Russia from the ninth century, later the Mongolians conquered Russia. After the ruling of the Mongolians the autocratic czars of Moscow came, they gained land by overpowering and expropriating the nobility and landlords. The czars reigned autocratic, which in theory means that they were almighty. In reality it turned out that they were unable to achieve a lot in such an overwhelming big country.

\subsection{A nation of many peoples, instead of one common nation}

Through great territorial expansion, Russia became a nation with a lot of different peoples. There was not really a common identity. Of course in different parts of Russia, groups would identify themselves as one people. For example in the European part of Russia, many people identified themselves as Russians. But there was no \quote{people of Russia} in the whole \quantity{17.098.242}{km^2} that belongs to the nation Russia.

\subsection{Slaves and servants instead of civilians}

During the regime of the czars, the boyars (nobility) became servants of the czar: they got their land in exchange for absolute submission. The farmers became slaves, the church became the church of the state: the czars reigned autocratic. The relationship between the czar and the boyars differed from the relationship the European kings had with there nobility. In most European kingdoms, the nobility and knights had their own land, their own districts which they reigned. They were more or less on equal footing with their king.

\subsection{Intelligentsia instead of middle class}

This submission meant that there was no public discussion. Participation in the national decisions could only be done through connections with the czar and his advisers. A kind of opposition or protest came from the intelligentsia. The intelligentsia is a social class of people who think wisely and creatively about the development and spreading of culture. It existed mainly of teachers, engineers and academic people. They thought about relevant questions and had their own opinions, but the only way to express this opinion, to give it publicity, was by protests, revolutions.

\subsection{Revolution instead of equality}

Because there was no public discussion, there was no smooth development or modification initiated by the people from below. There was just forced growth, forced development induced by the government. This had consequences such as revolutions from the revolutionary intelligentsia.

\subsection{Totalitarism instead of pluralism}

The lack of equality and public discussion led to the Russian revolution, a combination of total war and extreme ideologies. The communists wanted to create a new society with force, with weapons, with war. Communism (in Russia) was totalitarian; again society was brought to submission.

\subsection{Expansion, but no development}

In Soviet Russia, Stalins aim was to develop from a retarded country to a developed country: collectivization, industrialization, a planned economy. But this all was forced by the government. The people had no intrinsic motivation.

\subsection{Transition instead of westernizing}

Communism developed a fatal gap between the political party and the people. The perestrojka, written to renew the stagnated development of the Soviet Union, turned out to be a katastrojka: the Sovjet Union fell apart. Putin built a new nation: Russia. Russia became a modern society, but with its own values.

In summary, Henk Kern accentuated the differences between the history of Russia and Europe. The difference in the relationship between the czar and his landlords compared to the relationship between European kings (i.e. King Louis) and their knights. The difference between a huge nation with lots of peoples and a continent divided into lots of different nations, each with its own people. The difference between the Russian autocracy and the European kingdoms. The difference between a nation which admits the opinion of its people and a nation which forbids having an own opinion.

\section{The Second lecture}

\placemarginfigure
  {Prof.dr. W. van den Bercken}
  {\externalfigure[wilvdbercken][width=\marginwidth]}

Professor Wil van den Bercken, professor in the Christianity of Russia and the Ukraine at our Radboud University in Nijmegen, told us about the architecture, mainly of (orthodox) churches in Moscow en Saint Petersburg. Where Henk Kern accentuated the differences between Russia and Europe, Prof. van de Bercken underlined the similarities. The Russian architecture of churches and palaces is strongly influenced by the (south) European architecture. He started with Moscow, where, like he says, his heart lays. Moscow is a vivid city, a city with history, with a soul, but also busy and chaotic. Whereas Saint Petersburg is built in very short time, on the order of Peter the Great in the bog. In no time there arose a planned, structured city, without heart, but with beautiful buildings. Actually all buildings are palaces. Saint Petersburg is esthetically wonderful.

Despite of fundamental differences in history and politics, Russia has a European culture. Sometimes Moscow is called the third Rome. After the real Rome, with its emperors and Constantinople in the Byzantine Empire, Moscow felt itself the political and spiritual successor of the Byzantine Empire. This illustrates the European roots of the Russian culture. In Europe there always was a fascination for Rome, for its emperors, for its art and culture.

\subsection{Moscow}

A lot of churches, mainly on the \emph{Kremlin}, are designed by Italian architects. Often you can clearly see the European influences. The \emph{Cathedral of the archangel Michael} in the Moscow Kremlin, which is the burial church of the czars, has influences from both cultures: the squared ground plan, based on a Greek cross\footnote{A Greek cross is a cross with four equal sized arms. This in contrast to a Latin cross, which we know from the protestant and catholic church, which has two equal sized horizontal arms, but vertically the undermost arm is substantially larger than the up most and the horizontal arms.} is clearly a Byzantine influence, whereas the shell formed roof arcs and the stonework on the walls are in (European) Renaissance style. It was built by Italian architects at the end of the 15th century.

\placefigure
  {Some cathedrals in Moskow}
  \startcombination[3*1]
    {\externalfigure[archangelcath]   [height=0.2\textheight]} {Cathedral of the Archangel Michael}
    {\externalfigure[annunciationcath][height=0.2\textheight]} {Cathedral of Annunciation}
    {\externalfigure[assumptioncath]  [height=0.2\textheight]} {Cathedral of Assumption}
  \stopcombination

Another church in the Kremlin is the \emph{Cathedral of the Annunciation}. This is more a typically Russian church: simple, with golden union domes\footnote{An onion dome is a dome whose shape resembles the onion.}. Inside you can clearly see the catholic influences: a lot of images and icons of saints.

The  \emph{Assumption Cathedral} is the main church of the patriarch\footnote{A patriarch is the religious head of a patriarchy, a big religious region. The patriarch settled in Moscow is patriarch of the entire Russian church.}. This church is 100\% in Russian style although it is built by Italian architects.

A lot of buildings in the Kremlin are designed by Italian architects, but they tried to implement the Russian culture. Therefore the Kremlin symbolizes two things: on one hand it symbolizes the Russian identity. On the other hand it is a perfect example of European architecture in Russia.

A really Russian element of the churches is the atmosphere inside. Looking at the theology and liturgy of the catholic and the Orthodox Church, there are many similarities. But in nearly everything, the Orthodox Church is more pompous, more impressive. The essential difference between the two is that the Orthodox Church does recognize the pope. Each district has its own patriarch. The atmosphere in Russian churches is totally different from that in the churches we know and have visited in Europe. Inside Russian orthodox churches the walls are covered with frescos, icons, paintings and mosaics. The priests are dressed in overwhelming gold brocade, a smell of incense completes the simulation of heaven on earth.

\subsection{Saint Petersburg}

\placefigure[right]
  {Some cathedrals in Saint Petersburg}
  \startcombination[1*2]
    {\externalfigure[peterandpaulcath][width=0.25\textwidth]} {Petrus and Paulus Cathedral}
    {\externalfigure[isaaccath]       [width=0.25\textwidth]} {Isaac Cathedral}
  \stopcombination

Saint Peterburg is built in the mud, in the delta of the Neva. Peter the Great came, ordered to build Saint Petersburg on that spot, and Saint Petersburg was built there! Builders sacrificed lots of slaves, struggled against the water, the midges, the weather and the extreme cold. But, since Peter the Great founded Saint Petersburg, Russia became an important nation in Europe, both in political issues as in cultural ways.

The architecture in Saint Petersburg is strongly influenced by European styles and therefore the city has a very European nature. The map of the city is like the one of Amsterdam, with canals which form semicircles. From the \emph{Petrus and Paulus Cathedral}, built by Peter the Great on the fortress, we see that they not only wanted to equalize the European architecture, but even exceed it. The Petrus and Paulus Cathedral had the highest tower of European churches.

The \emph{Isaac Cathedral} is on one hand typically Russian because of its squared floor plan. But on the other hand it is quite European with its dome\footnote{Each orthodox church has at least one dome to symbolize the heaven on earth.} and tympana.

At the end of this lecture, Wil and Henk went with us to the room of the student-association to drink a beer. This was a nice ending of this informative afternoon. I was curious what we would recognize from their stories in Moscow and Saint Petersburg. For this, take a look at the reports about Saint Peterburg, by Jasper Schadron, Maarten van de Griend, Tim Steenvoorden and Jorrit van de Boogaard and the Moscow reports by Martijn Jongen, Eveline Gieles, Frederik Kerling, Jeroen Meidam, Martijn Jongen and Madelon Bours.

\stopcomponent
