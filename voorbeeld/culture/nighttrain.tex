\startcomponent nightttrain
\environment    travellayout

\startchapter[title=The night train,
              label=cul:nachttrein]
             [date={\date[d=13,m=10,y=2009]},
              author=Sjoerd Verhagen,
              publications=no]

Our night train from Saint Petersburg to Moscow: in Dutch it would be discribed as a \quote{denderende reis}, in English it takes a few more words.

We were informed to gather in the hostel stricktly at 20:45~\PM. Prior to this departure, I interviewed some other participants on what would be their very first long train experience.

\section{Interviews}

\subsection{Madelon}

\startlines
\emph{Have you ever made such a long journey by train?}
\quote{Never.}
\emph{Do you expect a good night's rest?}
\quote{I am planning to get to bed early to make sure I'm fit to get up before arrival (6:00\AM)}
\emph{Anything else you are hoping for?}
\quote{Seeing the milky way nice and bright would be terrific, in the region between Saint Petersburg and Moscow, light pollution is far lass than in the Netherlands.}
\stoplines

\subsection{Martijn}

\startlines
\emph{Have you ever made such a long journey by train?}
\quote{No, never.}
\emph{What about this trip are you looking foreward to?}
\quote{I'm looking foreward to drinking wodka in the cabin because there won't be nothing to do at all.}
\emph{Is there something you are afraid of about the trip?}
\quote{It would really set me up if everyone just went to sleep right away and leave me and my wodka all alone. Also Russians, I would hate to share the sleeping cabin with someone I don't know and cannot speak with.}
\stoplines

\subsection{Tim}

\startlines
\emph{Have you ever made such a long journey by train?}
\quote{Yes I have, a trip from Amsterdam to Poznań, Poland. It was very relaxing back then so I have good hopes for tonight.}
\emph{Imagine there shines a beautiful full moon during the night, may we wake you up?}
\quote{No, I prefer sleeping.}
\stoplines

\subsection{Jeroen}

\startlines
\emph{Have you ever made such a long journey by train?}
\quote{Nope.}
\emph{What do you like about this trip?}
\quote{I like to draw trains, preferably electric ones but these diesels will also be cool. So taking good notice of these machines will allow a lot of fun back home.}
\emph{Would you like to have one of the upper of one of the lower beds?}
\quote{I don't know how the cabins look like so I don't care as long as there is enough room for my luggage.}
\stoplines

\subsection{Jasper}

\startlines
\emph{Have you ever made such a long journey by train?}
\quote{Never, unfortunately.}
\emph{Could we wake you up if a really cosy faintly lighted little town comes by during the night?}
\quote{Yes! I wouldn't want to miss that for the world.}
\emph{Any fears for the night?}
\quote{Having trouble getting some good rest or being woken up for no good reason.}
\stoplines

\subsection{Maarten}

\startlines
\emph{Have you ever made such a long journey by train?}
\quote{Contrary to most people, I haven't ever slept in a train anyhow.}
\emph{Is there something you are looking foreward to?}
\quote{Drinking wodka with Martijn!}
\emph{Is there something you are afraid of?}
\quote{At first this whole night train idea put me off a bit, but now we are about to go, I'm looking foreward to it. As long as there are no Russians in my cabin, I will have a good time.}
\stoplines

\subsection{Andrei}

\startlines
\emph{Have you ever made such a long journey by train?}
\quote{Yes, multiple times. The longest trip was from Bellarussia to Duisburg, it wasn't that nice but it brought me where I needed to be.}
\emph{What if you, our Russian speaking guide, were to share your cabin with some Russians?}
\quote{It would be fine, as long as they allowed me to sleep.}
\stoplines

\subsection{Guus}

\startlines
\emph{Have you ever made such a long journey by train?}
\quote{Nope.}
\emph{You are a organiser of this tour through Russia, why are we taking the nighttrain as opposed to flying?}
\quote{Early on in the organising process we decided on taking a nighttrain instead of flying. It does not comprise as much paperwork and troubles as handeling the airport process. Besides, if we sleep during traveling, our day program can contain more science, not to mention the saving of one hostelnight.}
\emph{How do you feel about getting up before 6:00\AM?}
\quote{I hate to get up in the dark, but there is no choice. Luckily my associates will handle the process of unboarding the train and reaching our hostel.}
\stoplines

\subsection{Rianne}

\startlines
\emph{Have you ever made such a long journey by train?}
\quote{Yes, Eindhoven to Austria for holidays.}
\emph{Martijn is looking more and more thirsty by the minute, would you join him for a wodka on the tracks?}
\quote{I sure would!}
\emph{Does something about tonight trouble you?}
\quote{Sharing a small cabin with strangers I cannot communicate with in the dark night gives me shivers. The rest of me is very excited.}
\stoplines

\subsection{Frederik}

\startlines
\emph{Have you ever made such a long journey by train?}
\quote{Nope.}
\emph{Wodka on the tracks?}
\quote{Yes please.}
\emph{How do you feel about this journey?}
\quote{I'd rather have a normal flight but it's not like the train bothers me.}
\emph{For what view outside or event in the train should we wake you up?}
\quote{If suspicious things happen I want to know, I am not going to be robbed on this train. I am more than happy with the police presence onboard.}
\emph{Anything you like to let us know?}
\quote{Tomorrow I am going to be a zombie, do not talk to me, do not expect from me more than the strictly nessecary.}
\stoplines

\subsection{Jonas}

\startlines
\emph{Have you ever made such a long journey by train?}
\quote{Yes, it was a trip from Slovenia to Venice. I had to wait 12 hours on the station for the train was delayed, things like these shape a man.}
\emph{Wodka on the tracks?}
\quote{Yes please, it helps me get to sleep, plus we could meet up with some locals and improve our Russian!}
\emph{In what case should we wake you up?}
\quote{In case of danger, and for danger only.}
\emph{Are you afraid of the early rise and shine?}
\quote{No I am not, I can handle getting up early, contrary to people who whine.}
\stoplines

\subsection{Ans}

\startlines
\emph{Have you ever made such a long journey by train?}
\quote{Yes, from Athene to Tesseloniki, I traveled with a friend so we shared the cabin with two strangers. They went to sleep right away so we did not have a lot of contact.}
\emph{Care for a nighttime wodka? I heard it helps you get to sleep.}
\quote{Yes please! I know what you are going to ask next: you may wake me up for the full moon, but only with a song.}
\stoplines

\section{The nighttrain}

\placefigure[right]
  {Our train}
  {\externalfigure[train][small]}

At 20:52~\PM\ it was time to say goodbye to our beloved Crazy Duck hostel.

In the metro, we lost Andrei and Martijn after we got out of a train. Many next trains passed but the two were not to be seen. As telephone reception is nonexistent that deep underground, radiocontact was not possible. Confident of Andrei's capabilities, and remembering the 10\% measurement uncertainty on participant number allowed we decided to move on. A happy reuniting happened on the next and last metro station, the two had gone on alone assuming we had done the same.

After our recon unit (Margot \& Andrei) found the correct train, we all moved into waiting position. The trains still had their communist logos in place and were a sight to see. We were informed to all stay in our cabins for the first half our to allow the organisers to hand out paperwork.

\placefigure[left]
  {Training at night, no worries!}
  {\externalfigure[coupe][small]}

Your author shared a cabin with Guus, Maarten and Jorrit, deeply unsatisfying compared to Andrei, who was allowed to spend the night with no less than three of our girls. The cabins were capable of housing all our luggage and all nessecary sleeping equipment was present. A quick survey of the train revealed Jins, Frank, Jeroen and Martijn as the most notorious drinkers, a whopping three bottles of Wodka could be found on their humble cabin table. Four of our girls, Ans, Eveline, Madelon and Inka shared a room together. It's just neccesary, Eveline explained.

Curiously the train speakers played sad music upon departing the Moskova station\footnote{Which is indeed situated in Saint Petersburg, the former Leningrad.} for our destination: Leningrada station\footnote{Which is indeed situated in Moscow.}. Departure was at 22:22~\PM\ exactly, as planned by the way. Guus notified us that he has \quote{accidently} stolen a towel from our beloved hostel, an event which would repeat itself further on in our Russian journey.

We managed to place everyone in a single cabin, which is a very cosy experience, if you are not claustrophobic. Here, Jonas and your author are awarded a bottle of wodka for posing the most interesting question in the recent scientific visits.

\placefigure[left]
  {Claustrophobia!}
  {\externalfigure[claustrophobia][medium]}

While Jasper and Maarten proceed to play some chess and complain about the music which can't be turned off, Jonas meets up with a Russian technical student: Vova, or perhaps Wowa. He does not speak much of English but his jar of home made vodka gives us a good hint about his intentions. His mother passes our cabin and warns her son to watch out with that wodka; so much for the concealing strategy. The conversation subject is hard to elevate above girls and vodka. After we succeed in declining any more than would be lethal amount of his vodka\footnote{Which we drank from empty tea glasses.}, we decide to call in our Russian speaking Andrei to politely ask the young man to allow us some sleep. The poor chap did not seal his jar of wodka well enough upon leaving, he accidentally poured half of it in his bag. Luckily the sugarless wodka does not stick.

\placefigure[right]
  {\date[d=27,m=11,y=2009]\ Train accident on our tracks. I would like to thank the organizing committee for their well chosen dates of travelling.}
  {\externalfigure[ongeluk][small]}

The train is so long the locomotive cannot be heard. Combining this relative silence with an ambient temperature of 20\degree C results in a good night's rest on the part of the author from 0:00\AM\ till 5:30\AM. At least that was what we planned. Half an hour of time to get ready for leaving the train was not enough in the eyes of our provodnika. At 5:00\AM\ she came to our doors to protrude Martijns dream of a black man with salmon. Fortunately a collective quick \quote{da} from our beds made her leave. Waking up was harsh, buying sweet \quote{empire} tea from the conductor eased our pains. Only one minute late, at 6:03\AM\ we arrive at Leningrada station, Moscow.

A quick survey among my fellow travellers afterwards tells a train of equal flamboyance for the transsiberia trip would be satisfying.

\stopchapter

\stopcomponent
