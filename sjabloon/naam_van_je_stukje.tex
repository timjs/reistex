\startcomponent naam_van_je_stukje
\environment    travellayout

\startchapter[title=Titel van je stukje]
             [author=Jouw Naam,
              date={\date[d=1,m=1,y=2011]}]

\section{Inleiding}

Dit is een sjabloon dat je kunt gebruiken om een stukje te schijven voor het reisverslag of de reader. Vergeet niet:

\startitemize
\item De titel, datum en naam aan te passen in het \type{\startchapter}-commando.
\item De \type{\startcomponent}-, \type{\stopcomponent}-, \type{\stopchapter} en \type{\environment}-regels \emph{nooit} te verwijderen!
\item Dit bestand altijd op te slaan in \emph{Unicode/UTF-8} codering.
\item Geen spaties te gebruiken in je bestandsnaam.
\item Er voor te zorgen dat de naam achter \type{\startcomponent} overeenkomt met je bestandsnaam (zonder extensie) \emph{en} de naam van je bibliografie.
\stopitemize

De rest van dit document bevat voorbeelden van commando's die je kunt gebruiken.\footnote{De meeste spreken voor zich.} Dit hoef je natuurlijk niet in je stukje te laten staan. Meer informatie over de commando's in dit sjabloon kun je vinden in de handleiding.

\section{Knippen en plakken}

\subsection{Tekst}

% Regels die beginnen met een procent-teken zijn commentaar en komen niet in je stukje terecht.

\emph{Benadrukken}, \cap{KAPitalen}, \quote{aanhaling met daarbinnen een \quote{aanhaling}}, \high{hoge} en \low{lage} tekst, voetnoot\footnote{Komt onder aan de pagina.}, iets langer streepje --, puntjes\dots

Alinea's worden gescheiden door een witregel. Špéçiåłe tëkèñß kün jê éénvœđïg zø ĭnŧıkkēn. Dit kan met het hele αλφαβητος, maar ook vóór bedragen €5,42. Let verder op \%, \#, \$, \&, \{, \}, \~, en \letterbackslash.

\subsection{Opsommingen}

\startitemize
\item puntje
\item ander puntje
\stopitemize

\startitemize[n]
\item een
\item twee
\item etc.
\stopitemize

\startitemize
\head Term

Omschrijving

\head Andere term

Andere omschrijving
\stopitemize

\subsection{Figuren}

Het figuur met een label is \in{figuur}[fig:label], hij staat op \at{pagina}[fig:label].

\placefigure[][fig:label]
  {Figuur met een label zodat je er naar kunt verwijzen.}
  {\externalfigure[bestandsnaam_zonder_extensie][large]}

\placefigure
  {Figuur zonder label.}
  {\externalfigure[andere_bestandsnaam_zonder_extensie][small]}

\subsection{Tabellen}

Verwijzen gaat net zo als bij figuren en formules (zie \in{tabel}[tab:nummers]).

\placetable[][tab:nummers]
  {Even en oneven getallen}
  \starttabulate[|c|c|]
  \HL
  \NC  even  \NC  oneven  \NR
  \HL
  \NC  0     \NC  1       \NR
  \NC  2     \NC  3       \NR
  \NC  4     \NC  5       \NR
  \NC  6     \NC  7       \NR
  \NC  8     \NC  9       \NR
  \HL
  \stoptabulate

\subsection{Formules}

\placeformula[for:schrödinger]
\startformula
i\hbar \frac{\partial}{\partial t} \Psi
= \left( -\frac{\hbar^2}{2m} \nabla^2 + \hat{V} \right) \Psi
\stopformula

De Schrödingervergelijking staat in \in{formule}[for:schrödinger]. Tijdsonafhankelijk is dit:
\startformula
E \psi = \left( -\frac{\hbar^2}{2m} \nabla^2 + \hat{V} \right) \psi.
\stopformula

Dit kun je natuurlijk ook schrijven als $\hat{H}\psi=E\psi$ \cite[Schrodinger:1926].

\subsection{Afkortingen}

\abbreviation{EM}{Elektromagnetisme}

\EM\ beschrijft de fysica van het elektrische- en het magnetische veld. Dit staat voor \infull{EM} \cite[Griffiths:2008].

\subsection{Eenheden}

\startfact
\fact plaats      \\ x \\ \quantity{13,37}{m}     \\
\fact snelheid    \\ v \\ \quantity{10}{m/s}      \\
\fact versnelling \\ a \\ \quantity{-3e-3}{m/s^2} \\
\fact tijd        \\ t \\ \quantity{42}{s}        \\
\stopfact

De elektronmassa wordt gegeven door $m_{\text{e}} = \quantity{0,510998910(13)}{MeV}$ \cite[Wikipedia:Electronmass].

\section{Compileren}

Je kunt je eigen stukje compileren met het commando:

\starttyping
> texexec naam_van_je_stukje.tex
\stoptyping

Als je een readerstukje aan het maken bent, voeg dan \type{---mode=reader} (met twee streepjes!) aan het commando toe.
Je krijgt dan een mooi \PDF'je met al je tekst en plaatjes. Vergeet niet om alle plaatjes, je bibliografie en het bestand \type{travellayout.tex} in \emph{dezelfde map} te zetten.

\stopchapter

\stopcomponent
