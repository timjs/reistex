%D \section{Catcodes}
%D
%D We maken gebruik van een iets modernere \emph{catcode} instellingen.
%D Onderstaande is al standaard in Mark IV (de nieuwste versie van \CONTEXT).
%D Simpel gezegd houdt het in dat de tekens \type{^}, \type{_}, \type{&}
%D en \type{|} geen speciale betekenis meer hebben in de lopende tekst.
%D \type{^} en \type{_} kunnen natuurlijk nog steeds gebruikt worden in
%D wiskunde modus!

% Maakt alleen ^ en _ other:
%\nonknuthmode

% Zo doen we ook & en |:
%\define[1]\makecharacterother
%  {\catcode`#1\other}
%\makecharacterother _
%\makecharacterother ^
%\makecharacterother &
%\makecharacterother |
% Maar beiden gaan falicant mis na \getbuffer en wss ook andere commando's

% En deze pakt ie al helemaal niet goed:
%\startextendcatcodetable \ctxcatcodes
%    \catcode\underscoreasciicode\othercatcode
%    \catcode\circumflexasciicode\othercatcode
%    \catcode\ampersandasciicode \othercatcode
%    \catcode\barasciicode       \othercatcode
%\stopextendcatcodetable

% Is dit nodig?
%\let\\\letterbackslash
%\let\~\lettertilde
% Deze twee niet denk ik...
%\let\{\letteropenbrace
%\let\}\letterclosebrace


%D \section{Documentatie}
%D
%D Zie de handleiding.
%D
%D \section{Aanpassingen}
%D
%D \subsubsubject{juli – augustus 2010}
%D
%D Initiële versie.
%D
%D \startitemize
%D \item Opmaak zoals bij het oude reisverslag
%D \item Macro voor hoofdstukken met auteur en datum
%D \item \LATEX-compatibele macro's
%D \item Zijfiguren
%D \item Vele afkortingen en \cap{url}'s
%D \item Modes voor boekjes en correctieversies
%D \stopitemize
%D
%D Tim Steenvoorden
%D
%D \subsubject{oktober – november 2010}
%D
%D Versies 0.5 en 0.7
%D
%D \startitemize
%D \item Paginamarges
%D \item Microtyping
%D \item Onderlijnde koppen
%D \item Inhoudsopgave
%D \item Bibliografieën en referenties
%D \item Lijnen van tabellen en tabulaties
%D \item Opmaak koppen
%D \stopitemize
%D
%D Tim Steenvoorden
%D
%D \subsubject{januarie 2011}
%D
%D Versie 0.9
%D
%D \startitemize
%D \item Citaten
%D \item Sjabloondocument voor reisdeelnemers
%D \stopitemize
%D
%D Tim Steenvoorden
%D
%D \section{Te doen}
%D
%D \startitemize
%D \item Draaiboek
%D \item Documentatie uitbreiden
%D \item Samenvatting schrijven
%D \item \tex{E} toevoegen?
%D \stopitemize


\usetypescript[serif][hanging][normal]
\setupalign[hanging]



\setupsection[part]
  [conversion=Romannumerals]

%D \section{Te doen}

%FIXME: als register
\definelist[todo]
  [alternative=a,
   partnumber=no,
   pagestyle=\quad\it]
\definecombinedlist[todos]
  [chapter,todo]
\setupcombinedlist[todos]
  [level=current,% Waarom weet ik ook niet, maar het werkt.
   interaction=pagenumber]

\starttexdefinition todo #1
  \inmargin{Todo}
  \underbar{#1}
  \writetolist[todo]{}{#1}
\stoptexdefinition



[x] auto \placepublications aftersection chapter or redefine \completepublications?
[ ] rule before footnotes
[ ] distance before text floats around left/right figure
[ ] marginfigures
[ ] inmargin always in outer
[ ] redefine \cap?
[ ] space between number and heading in \subsection

[ ] titels secties Commando's en Schrijven
